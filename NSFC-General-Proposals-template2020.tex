%\documentclass[a4paper,11pt]{article}
%%%%%%%%%%%%%%%%%%%%%%%%%%%%%%%%%%%%%%%%%%%%%%%%%%%%%%%%%%%%%%%%%%%%%%%%%%%%%%%%%%%%%%%%%%
%% NSFC 2020  specific settings
%%%%%%%%%%%%%%%%%%%%%%%%%%%%%%%%%%%%%%%%%%%%%%%%%%%%%%%%%%%%%%%%%%%%%%%%%%%%%%%%%%%%%%%%%%
\documentclass[a4paper,12pt, AutoFakeBold, fontset=adobe]{ctexart}
\usepackage{xcolor}
\definecolor{nsfcblue}{RGB}{0,112,192}
\ctexset{
  section={
    format= \color{nsfcblue}\zihao{4}\raggedright\kaishu\bfseries,
    name = {(,)},
    number = \chinese{section}
  },
  subsection={
    format= \color{nsfcblue}\zihao{4}\raggedright\kaishu\bfseries,
    name = {,. },
    number = \arabic{subsection}
  },
  subsubsection={
    format= \color{nsfcblue}\zihao{4}\raggedright\kaishu\bfseries,
    name = {(,) },
    number = \arabic{subsubsection}
  },
}
\usepackage{fancyhdr}
\pagestyle{plain}
%%%%%%%%%%%%%%%%%%%%%%%%%%%%%%%%%%%%%%%%%%%%%%%%%%%%%%%%%%%%%%%%%%%%%%%%%%%%%%%%%%%%%%%%%%

\usepackage{amsmath}
\usepackage{bm}

\usepackage{natbib}

%% Reduce Bibliography space
\setlength{\bibsep}{0.0pt}
\setlength{\itemsep}{0.0pt}

%% Remove item space
\usepackage{enumitem}
%\setlist[itemize]{noitemsep,nolistsep}
\setlist[enumerate]{noitemsep,nolistsep}


\usepackage{url}
\usepackage[colorlinks=true,citecolor=blue,linkcolor=blue]{hyperref}
\usepackage[margin=0.85in]{geometry}

\usepackage{booktabs}
\usepackage{setspace}

%% Remove page numbers (and reset to 1)
\pagenumbering{gobble}

\title{\bfseries\kaishu\zihao{3} 报告正文}
\author{\vspace{-2cm}}
\date{\vspace{-2cm}}

\begin{document}
\maketitle

\section{立项依据与研究内容(建议8000字以下):}

\subsection{项目的立项依据}{\color{nsfcblue}\kaishu\noindent(研究意义、国内外研究现状及发
  展动态分析,需结合科学研究发展趋势来论述科学意义;或结合国民经济和社会发展中迫切需要解决
  的关键科技问题来论述其应用前景。附主要参考文献目录)}



\subsubsection{研究意义}

\subsubsection{国内外研究现状及发展动态分析}

\renewcommand\refname{\subsubsection{主要参考文献}}
\begin{spacing}{0.9}
\bibliographystyle{agsm}
\bibliography{full,References}
\end{spacing}


\subsection{拟采取的研究方案及可行性分析}{\color{nsfcblue}\kaishu\noindent(包括研究方法、
  技术路线、实验手段、关键技术等说明)}


\subsubsection{研究方法}

\subsubsection{技术路线}

\subsubsection{实验手段}

\subsubsection{关键技术}

\subsection{本项目的特色与创新之处}


\subsection{年度研究计划及预期研究结果}{\color{nsfcblue}\kaishu\noindent(包括拟组织的重要
  学术交流活动、国际合作与交流计划等)。}


\subsubsection{年度研究计划}

\subsubsection{预期研究结果}

\section{研究基础与工作条件}

\subsection{研究基础}{\color{nsfcblue}\kaishu\noindent(与本项目相关的研究工作积累和已取得
  的研究工作成绩)}

\subsection{工作条件}{\color{nsfcblue}\kaishu\noindent(包括已具备的实验条件,尚缺少的实验
  条件和拟解决的途径,包括利用国家实验室、国家重点实验室和部门重点实验室等研究基地的计划与
  落实情况)}

\subsection{正在承担的与本项目相关的科研项目情况}{\color{nsfcblue}\kaishu\noindent(申请人
  和项目组主要参与者正在承担的与本项目相关的科研项目情况,包括国家自然科学基金的项目和国家
  其他科技计划项目,要注明项目的名称和编号、经费来源、起止年月、与本项目的关系及负责的内容
  等)}

\subsection{完成国家自然科学基金项目情况}{\color{nsfcblue}\kaishu\noindent(对申请人负责的
  前一个已结题科学基金项目(项目名称及批准号)完成情况、后续研究进展及与本申请项目的关系加
  以详细说明。另附该已结题项目研究工作总结摘要(限500字)和相关成果的详细目录)}

\section{其他需要说明的问题}


\subsection{申请人同年申请不同类型的国家自然科学基金项目情
  况}{\color{nsfcblue}\kaishu\noindent(列明同年申请的其他项目的项目类型、项目名称信息,并
  说明与本项目之间的区别与联系)}

\subsection{具有高级专业技术职务(职称)的申请人或者主要参与者是否存在同年申请或者参与申请
  国家自然科学基金项目的单位不一致的情况;}{\color{nsfcblue}\kaishu 如存在上述情况,列明所
  涉及人员的姓名,申请或参与申请的其他项目的项目类型、项目名称、单位名称、上述人员在该项目
  中是申请人还是参与者,并说明单位不一致原因。}

\subsection{具有高级专业技术职务(职称)的申请人或者主要参与者是否存在与正在承担的国家自然
  科学基金项目的单位不一致的情况;}{\color{nsfcblue}\kaishu 如存在上述情况,列明所涉及人员
  的姓名,正在承担项目的批准号、项目类型、项目名称、单位名称、起止年月,并说明单位不一致原
  因。}

\subsection{其他。}


\end{document}
%%% Local Variables:
%%% TeX-engine: xetex
%%% End:
